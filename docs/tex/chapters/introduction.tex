\chapter*{Introduction}

In the modern society, the amount of information is far behind what one can remember or even process. In understanding this, one realizes the integral importance of delegation of thinking and information processing in a group. Decision making in groups and teams has been covered in the literature to some extent \cite{Black:groupdecisionmaking}. Supposing we have limited memory capacity, information required in decision making has to be distributed and communicated between people efficiently.

Our decisions can be either conscious or subliminal, depending on our needs or drives - whilst the former is connected with human behaviour, the latter is used for plausible agents. Just like in microeconomics, we can use utility as a measure of relative satisfaction \cite{Varian:micro} and see how one manages to fulfil their needs. In satisfying these needs, the knowledge stored in our memory and updated regularly is a key tool. With infinite memory, problems of information storage would be eliminated and with necessary information available at all times, provided it had once been acquired. Our memory, however, is limited. 

What we mean by saying that our memory is limited is that we are not able to remember everything. Certain pieces of information are fading away with time or as one is learning new facts. We set out to understand whether and how intensive communication can substitute insufficient memory space at a constant level of utility. It is obvious that adding the ability of communication improves the agents' chances to survive in the environment.

The goal of my work is to observe efficiency of the agents in their struggle to fulfill their needs using different implementations of spatial memory. Such agents will also be able to communicate with each others and thus possibly improve their chances.

This thesis consists of six parts. First, we will introduce the \emph{agent} and possible memory implementations based on specific examples \emph{(Chapter 1)}. Then I will explain algorithms to be used in the program, such as \emph{growing neural gas} \emph{(Chapter 2)}. In \emph{Chapter 3} We will describe the simulation, the agents, their memory and their communication. We will further describe the concrete implementation of the introduced algorithms in \emph{Chapter 4}. All experiments are about to be presented in \emph{Chapter 5} and the outcomes will be discussed in \emph{Chapter 6}.