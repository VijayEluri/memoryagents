\chapter{Related work}

\section{Agents}

There are several ways how to explain what or who the agent is. Apart from systems of agents used in philosophy or sociology, we can see a first modern use of agency and agents in economy where economists have substited the human with a simpler agent. They intended to simplify their economic models to be able to actually simulate something. Buyers and sellers are typical examples of agents used in simplified market model in microeconomics (see []). In this context agents are entities in the model which can act based on situation in the model.

For area of artificial intelligence we can use the definition of an agent which can be found in \cite{russel2003ai}. It cannot be more simple:

\begin{definition}{\bf Agent} is just something that acts.
\end{definition} 

Of course it is as general as it could be and for my purposes it is too simple, so I will use another definition which meets better the context of my work.

\begin{definition}{\bf Agent} is something that senses the environment and affects it using its actuators.
\end{definition} 

Having that defined we continue to specific kinds of agent

A rational agent refers back to economics where we can find a definition of rational behaviour. Even though it is rather a hypothetical model, as people are usually irrational in their decisions from the economics perspective, their is yet nice definition whereby a rational agent acts as if balancing costs against benefits to arrive at action that maximizes personal advantage (Milton Friedman (1953), Essays in Positive Economics).

A plausible agent
 
A narrative agent



\section{Spatial resource-bounded memory}

\cite{Ho:memoryarchitectures}