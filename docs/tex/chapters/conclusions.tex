\chapter{Conclusions}

I have created a multi-agent simulation with four different kinds of agents each of which differs in their approach to fulfill their needs. What they had to succeed in was they were put inside a two dimensional environment whereby they had to learn positions of six food resource so as to be able to survive. They were pure reactive agent, random agent, GNG agent and grid agent. Latter two had a memory to learn those positions. GNG agent used implemantion of growing neural gas, an unsupervised neural network, and grid agent used data structure inspired by \cite{Brom:placeandobjects}.

In the experiments I have compared those agents in their efficiency. First I have set up environments in which was only a single kind of agent and after that I have combined different kinds of agents together. The results of their efficiency measured by the overall hunger were presented in graphs and statistical variables such as mean, median, maximal and minimal value. 


 